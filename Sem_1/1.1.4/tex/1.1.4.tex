\documentclass[12pt,a4paper]{scrartcl}
\usepackage[utf8]{inputenc}
\usepackage[english,russian]{babel}
\usepackage{indentfirst}
\usepackage{misccorr}
\usepackage{graphicx}
\usepackage{amsmath}
\usepackage{multirow}
\usepackage{pgfplots}
\usepackage[top=1cm, bottom=1cm, left=1cm, right=1cm]{geometry}
\pgfplotsset{compat=1.9}

\begin{document}
	\graphicspath{{C:/Users/Alex/OneDrive/Изображения/TexImgs}}
	
	\newcommand{\ms}{\mathstrut}
	\newcommand{\msp}{\hspace{0.5cm}}
	\newcommand{\al}{\alpha}
	\newcommand{\dg}{^\circ}
	\newcommand{\qd}[2]{^{\frac{#1}{#2}}}
	\newcommand{\qdm}[2]{^{-\frac{#1}{#2}}}
	\newcommand{\lm}[2]{\underset{#1 \rightarrow #2}{\lim}}
	\newcommand{\sfrac}[2]{\dfrac{\strut #1}{\strut #2}}
	\newcommand{\equal}[1]{\overset{(#1)}{=}}
	\newcommand{\linevdots}{\ \raisebox{-.08\height}{\vdots}\ }
	\newcommand{\linecvdots}{\ \raisebox{-.08\height}{\vdots}\hspace{-0.13cm}\raisebox{.15\height}{\cancel{\phantom{a}}\hspace{0.06cm}}}
	\newcommand{\combox}[1]{\ms \msp \msp \begin{minipage}{0.95\linewidth}
			#1
	\end{minipage}}
	
	\newtheorem{pr}{Задача}
	\newtheorem{ex}{Пример}
	
	\newenvironment{slv}{\ms \msp \textit{Решение:}}{}
	\newenvironment{proof}{\ms \msp \textit{Доказательство: }}{\hfill $\square$}
	
	\begin{titlepage}
		
		\vspace*{\fill}
		
		\begin{center}
			\includegraphics[scale=0.8]{MIPT.png}
			\\[0.7cm]\Huge Московский Физико-Технический Институт\\(национальный исследовательский университет)
			\\[2cm]\LARGE Отчет по эксперименту
			\\[0.5cm]\noindent\rule{\textwidth}{1pt}
			\\\Huge\textbf{Измерение интенсивности радиационного\\фона}
			\\[-0.5cm]\noindent\rule{\textwidth}{1pt}
		\end{center}
		
		\begin{flushleft}
			\textit{Работа №1.1.4; дата: 06.09.21}\hfill\textit{Семестр: 1}
		\end{flushleft}
		
		\vspace*{\fill}
		
		\begin{flushleft}
			Выполнил: \hspace{\fill} Группа:
			\\Кошелев Александр \hspace{\fill} Б05-105
		\end{flushleft}
	\end{titlepage}
	
	%Страница 2
	
	\begin{flushleft}
		\footnotesize{Измерение интенсивности радиационного фона} \hspace{\fill} \footnotesize{2}
		\\[-0.3cm]\noindent\rule{\textwidth}{0.3pt}
	\end{flushleft}

	\section{Аннотация}
	\textbf{Цель работы:} применение методов обработки экспериментальных данных для изучения статистических закономерностей при измерении интенсивности радиационного фона.
	\par\textbf{В работе используются:} счетчик Гейгера-Мюллера (СТС-6), блок питания, компьютер с интерфейсом связи со счетчиком. 
	
	\section{Теоретические сведения}
	Пусть $n_i$ - число срабатываний за один интервал измерений, $\overline{n}$ - среднее число срабатываний, \\$N$ - общее количество срабатываний, $k$ - индекс, причем для $\tau = 10\,$с - 1, $\tau = 20\,$с - 2, $\tau = 40\,$с - 3. Для определения среднего числа срабатываний счетчика $\overline{n}_k$ будем использовать формулу:
	\begin{equation}
		\overline{n}_k = \sfrac{1}{N}\overset{N}{\underset{i = 1}{\sum}} n_i
	\end{equation}
	\par Среднеквадратичные ошибки отдельных измерений $\sigma_{k}$ определим как:
	\begin{equation}
		\sigma_k = \sqrt{\sfrac{1}{N-1}\overset{N}{\underset{i = 1}{\sum}}(n_i - \overline{n}_i)^2}
	\end{equation}
	\par Приблизительную среднеквадратичную ошибку обозначим как $\tilde{\sigma_{k}}$. Она определяется формулой:
	\begin{equation}
		\tilde{\sigma_{k}} = \sqrt{\overline{n}_k}
	\end{equation}
	\par Среднеквадратичную ошибку среднего $\sigma_{\overline{n}_k}$ вычислим по формуле:
	\begin{equation}
		\sigma_{\overline{n}_k} = \sqrt{\sfrac{1}{N(N-1)}\overset{N}{\underset{i = 1}{\sum}}(n_i - \overline{n}_i)^2}
	\end{equation}
	\par И, наконец, относительную ошибку измерений $\varepsilon_{\overline{n}_k}$ определим по формуле:
	\begin{equation}
		\varepsilon_{\overline{n}_k} = \sfrac{\sigma_{\overline{n}_k}}{\overline{n}_k}\cdot 100\%	
	\end{equation}

	\section{Проведение измерений и обработка данных}
	
	\paragraph{3.1 Установка} \hfill
	
	\par Установка представляет собой компьютер с интерфейсом связи со счетчиком. Счетчик Гейгера-Мюллера регистрирует количество частиц, проходящих через него.
	
	\paragraph{3.2 Качественный анализ полученных данных} \hfill
	
	\par 3.2.1 На основе графика, измеряемая величина флуктуирует.
	\par 3.2.2 Вначале среднее значение измеряемой величины сильно флуктуирует, но позже приходит к постоянному значению.
	\par 3.2.3 Флуктуации величины погрешности отдельного измерения уменьшаются, а сама она приходит к постоянному значению.
	\par 3.2.4 Флуктуации величины погрешности среднего значения измеряемой величины уменьшаются, как и сама эта величина.
	
	\newpage
	\begin{flushleft}
		\footnotesize{Измерение интенсивности радиационного фона} \hspace{\fill} \footnotesize{3}
		\\[-0.3cm]\noindent\rule{\textwidth}{0.3pt}
	\end{flushleft}
	
	\paragraph{3.3 Данные со счетчика} \hfill
	\par Составим таблицу данных, выданных компьютером для $N_1 = 200$ и $\tau = 20\,$с (Таблица 1).
	\begin{center}
		\begin{tabular}{|c|c|c|c|c|c|c|c|c|c|c|}
			\hline № опыта & 1 & 2 & 3 & 4 & 5 & 6 & 7 & 8 & 9 & 10
			\\\hline 0 & 21 & 24 & 24 & 26 & 25 & 25 & 32 & 19 & 19 & 25
			\\\hline 10 & 14 & 21 & 26 & 30 & 23 & 25 & 22 & 18 & 28 & 20
			\\\hline 20 & 26 & 22 & 30 & 25 & 29 & 27 & 28 & 23 & 16 & 27
			\\\hline 30 & 24 & 22 & 33 & 19 & 26 & 19 & 24 & 30 & 25 & 31
			\\\hline 40 & 26 & 21 & 22 & 14 & 26 & 26 & 27 & 32 & 31 & 22
			\\\hline 50 & 30 & 32 & 23 & 22 & 18 & 29 & 19 & 20 & 19 & 17
			\\\hline 60 & 16 & 19 & 31 & 31 & 25 & 21 & 23 & 29 & 20 & 25
			\\\hline 70 & 37 & 23 & 30 & 29 & 26 & 19 & 17 & 21 & 23 & 23
			\\\hline 80 & 23 & 16 & 31 & 29 & 29 & 21 & 15 & 26 & 24 & 21
			\\\hline 90 & 15 & 26 & 23 & 18 & 36 & 24 & 19 & 19 & 15 & 28
			\\\hline 100 & 28 & 23 & 19 & 27 & 8 & 17 & 15 & 33 & 29 & 24
			\\\hline 110 & 24 & 19 & 24 & 20 & 25 & 22 & 17 & 21 & 19 & 25
			\\\hline 120 & 27 & 24 & 28 & 20 & 24 & 30 & 25 & 16 & 20 & 20
			\\\hline 130 & 22 & 20 & 32 & 22 & 18 & 22 & 20 & 19 & 30 & 22
			\\\hline 140 & 27 & 19 & 27 & 18 & 22 & 29 & 32 & 23 & 28 & 24
			\\\hline 150 & 17 & 28 & 33 & 23 & 19 & 20 & 20 & 29 & 31 & 27
			\\\hline 160 & 21 & 15 & 30 & 32 & 21 & 25 & 20 & 11 & 26 & 20
			\\\hline 170 & 20 & 32 & 28 & 20 & 21 & 24 & 21 & 19 & 20 & 17
			\\\hline 180 & 27 & 16 & 28 & 27 & 16 & 30 & 24 & 21 & 26 & 32
			\\\hline 190 & 39 & 25 & 16 & 21 & 15 & 20 & 25 & 26 & 16 & 25
			\\\hline 
		\end{tabular}
		\\\textbf{Таблица 1.} Число срабатываний счетчика при $\tau = 20\,$с
	\end{center}
	\par Разобъем результаты таблицы 1 по двое, получим таблицу для $N_2 = 100$ и $\tau = 40\,$с (Таблица 2):
	\begin{center}
		\begin{tabular}{|c|c|c|c|c|c|c|c|c|c|c|}
			\hline № опыта & 1 & 2 & 3 & 4 & 5 & 6 & 7 & 8 & 9 & 10
			\\\hline 0 & 45 & 50 & 50 & 51 & 44 & 35 & 56 & 48 & 40 & 48
			\\\hline 10 & 48 & 55 & 56 & 51 & 43 & 46 & 52 & 45 & 54 & 56
			\\\hline 20 & 47 & 36 & 52 & 59 & 53 & 62 & 55 & 47 & 39 & 36
			\\\hline 30 & 35 & 62 & 46 & 52 & 45 & 60 & 59 & 45 & 38 & 46
			\\\hline 40 & 39 & 60 & 50 & 41 & 45 & 41 & 41 & 60 & 38 & 43
			\\\hline 50 & 51 & 56 & 25 & 48 & 53 & 43 & 44 & 47 & 38 & 44
			\\\hline 60 & 51 & 48 & 54 & 41 & 40 & 42 & 54 & 40 & 39 & 52
			\\\hline 70 & 46 & 35 & 51 & 55 & 52 & 45 & 56 & 39 & 49 & 58
			\\\hline 80 & 36 & 62 & 46 & 31 & 46 & 52 & 48 & 45 & 40 & 37
			\\\hline 90 & 43 & 55 & 46 & 45 & 58 & 64 & 37 & 35 & 51 & 41
			\\\hline
		\end{tabular}
		\\\textbf{Таблица 2.} Число срабатываний счетчика при $\tau = 40\,$с
	\end{center}
	
	\paragraph{3.4 Построение гистограмм} \hfill
	
	\par Приведем данные для построения гистограмм при интервалах $\tau = 10\,$с и $\tau = 40\,$с (Таблицы 3, 4).
	
	\newpage
	\begin{flushleft}
		\footnotesize{Измерение интенсивности радиационного фона} \hspace{\fill} \footnotesize{4}
		\\[-0.3cm]\noindent\rule{\textwidth}{0.3pt}
	\end{flushleft}
	
	\begin{center}
		\begin{tabular}{|c|c|c|c|c|c|c|c|}
			\hline Число импульсов $n_i$ & 4 & 5 & 6 & 7 & 8 & 9 & 10 
			\\\hline Число случаев & 1 & 2 & 12 & 18 & 35 & 47 & 40 
			\\\hline Доля случаев  $\omega_i$ & 0.0025 & 0.005 & 0.03 & 0.045 & 0.0875 & 0.1175 & 0.1
			\\\hline Число импульсов $n_i$ & 11 & 12 & 13 & 14 & 15 & 16 & 17
			\\\hline Число случаев & 45 & 48 & 40 & 26 & 30 & 18 & 12
			\\\hline Доля случаев  $\omega_i$ & 0.1125 & 0.12 & 0.1 & 0.065 & 0.075 & 0.045 & 0.03
			\\\hline Число импульсов $n_i$ & 18 & 19 & 20 & 22 & 23 & 24 & 25
			\\\hline Число случаев & 8 & 4 & 9 & 1 & 1 & 1 & 1
			\\\hline Доля случаев  $\omega_i$ & 0.02 & 0.01 & 0.0225 & 0.0025 & 0.0025 & 0.0025 & 0.0025
			\\\hline
		\end{tabular}
		\\\textbf{Таблица 3.} Данные для построения гистограммы распределения числа 
		\\срабатываний датчика при $\tau = 10\,$с
	\end{center}
	\begin{center}
		\begin{tabular}{|c|c|c|c|c|c|c|c|}
			\hline Число импульсов $n_i$ & 31 & 35 & 36 & 37 & 38 & 39 & 40 
			\\\hline Число случаев & 1 & 4 & 3 & 2 & 3 & 4 & 4 
			\\\hline Доля случаев  $\omega_i$ & 0.01 & 0.04 & 0.03 & 0.02 & 0.03 & 0.04 & 0.04
			\\\hline Число импульсов $n_i$ & 41 & 42 & 43 & 44 & 45 & 46 & 47
			\\\hline Число случаев & 5 & 1 & 4 & 3 & 8 & 7 & 3
			\\\hline Доля случаев  $\omega_i$ & 0.05 & 0.01 & 0.04 & 0.03 & 0.08 & 0.07 & 0.03
			\\\hline Число импульсов $n_i$ & 48 & 49 & 50 & 51 & 52 & 53 & 54
			\\\hline Число случаев & 6 & 1 & 3 & 6 & 6 & 2 & 3
			\\\hline Доля случаев  $\omega_i$ & 0.06 & 0.01 & 0.03 & 0.06 & 0.06 & 0.02 & 0.03
			\\\hline Число импульсов $n_i$ & 55 & 56 & 58 & 59 & 60 & 62 & 64
			\\\hline Число случаев & 4 & 5 & 2 & 2 & 3 & 3 & 5
			\\\hline Доля случаев  $\omega_i$ & 0.04 & 0.05 & 0.02 & 0.02 & 0.03 & 0.03 & 0.05
			\\\hline
		\end{tabular}
		\\\textbf{Таблица 4.} Данные для построения гистограммы распределения числа 
		\\срабатываний датчика при $\tau = 40\,$с
	\end{center}

	На основании полученных данных построим гистограммы:
	\begin{center}
		\begin{tikzpicture}
			\begin{axis}[
				ybar,
				width = 350pt,
				xlabel = {$\ \ \ \ \; \,  27 \; \ \ \ \ \ 32 \; \ \ \ \ \; 37 \; \ \ \ \ \; 42 \; \ \ \ \ \; 47 \; \ \ \ \ \; 52 \; \ \ \; \ \ 57 \; \ \ \ \ \ 62 \; \ \ \ \ \; 67$},
				/pgf/number format/1000 sep={},
				legend style={
					at={(0.85, 0.9)},
					anchor=south,
					legend columns=-1
				}
				]
				\addplot coordinates {
					(-10, 0) (4, 1) (5, 2) (6, 12) 
					(7, 18) (8, 35) (9, 47)
					(10, 40) (11, 45) (12, 48)
					(13, 40) (14, 26) (15, 30)
					(16, 18) (17, 12) (18, 8)
					(19, 4) (20, 9) (22, 1)
					(23, 1) (24, 1) (25, 1)
				};
				\addplot coordinates {
					(-6, 1) (-2, 4) 
					(-1, 3) (0, 2) (1, 3)
					(2, 4) (3, 4) (4, 5)
					(5, 1) (6, 4) (7, 3)
					(8, 8) (9, 7) (10, 3)
					(11, 6) (12, 1) (13, 3)
					(14, 6) (15, 6) (16, 2)
					(17, 3) (18, 4) (19, 5)
					(21, 2) (22, 2)
					(23, 3) (25, 3) (27, 5)
				};
				\legend{10 c, 40 c}
			\end{axis}
		\end{tikzpicture}
	\end{center}
	
	\newpage
	\begin{flushleft}
		\footnotesize{Измерение интенсивности радиационного фона} \hspace{\fill} \footnotesize{5}
		\\[-0.3cm]\noindent\rule{\textwidth}{0.3pt}
	\end{flushleft}
	
	
	\paragraph{3.5 Рассчет средних значений и ошибок} \hfill
	
	Воспользуемся формулой (1) для рассчета средних значений числа срабатываний, а также формулами (2-4) для ошибок. На их основе составим таблицу (Таблица 5).
	
	\begin{center}
		\begin{tabular}{|c|c|c|c|c|c|}
			\hline $k$ & $\overline{n}_k$ & $\sigma_{k}$ & $\tilde{\sigma_{k}}$ & $\sigma_{\overline{n}_k}$ & $\varepsilon_{\overline{n}_k}$, \%
			\\\hline 1 & 11.77 & 3.47 & 3.43 & 0.17 & 1.4
			\\\hline 2 & 23.47 & 5.18 & 4.84 & 0.37 & 1.6
			\\\hline 3 & 46.99 & 7.83 & 6.85 & 0.78 & 1.7
			\\\hline
		\end{tabular}
		\\\textbf{Таблица 5.} Ошибки и средние
	\end{center}

	Определим также долю случаев, когда отклонения не превышают $\sigma_k$ и $2\sigma_k$, результат занесем в таблицу 6.
	\begin{center}
		\begin{tabular}{|c|c|c|c|c|}
			\hline Среднее & Ошибка & Число случаев & Доля случаев, \% & Теоретическая оценка, \%
			\\\hline \multirow{2}{*}{$\overline{n}_1$ = 11.77} & $\pm \sigma_1$ = 3.47 & 276 & 69.0 & 68.0
			\\ & $\pm 2\sigma_1$ = 6.94 & 381 & 95.3 & 95.0
			\\\hline \multirow{2}{*}{$\overline{n}_2$ = 23.47} & $\pm \sigma_2$ = 5.18 & 133 & 66.5 & 68.0
			\\ & $\pm 2\sigma_2$ = 10.36 & 195 & 97.5 & 95.0
			\\\hline \multirow{2}{*}{$\overline{n}_3$ = 46.99} & $\pm \sigma_3$ = 7.83 & 78 & 78.0 & 68.0
			\\ & $\pm 2\sigma_3$ = 15.66 & 99 & 99.0 & 95.0
			\\\hline
		\end{tabular}
		\\\textbf{Таблица 6.} Процент попадания точек в промежуток среднего значения 
	\end{center}

	\section{Итоги эксперимента}
	В ходе эксперимента были получены такие значения количества частиц, создающих радиационный фон:
	$$n_1 = 11.77 \pm 0.17\ \text{при} \ \varepsilon_{n_1} = 1.4\%$$
	$$n_2 = 23.47 \pm 0.37\ \text{при} \ \varepsilon_{n_2} = 1.6\%$$
	$$n_1 = 46.99 \pm 0.78\ \text{при} \ \varepsilon_{n_3} = 1.7\%$$
	
	В соответствии с данными таблицы 5, среднеквадратичная ошибка отдельного измерения близка к теоретической оценке $\sqrt{n}$. Таким образом, данная оценка в самом деле верна для случайных процессов.
	
	На основании полученных гистограмм экспериментально подтверждено подчинение случайного процесса распределению Гаусса. Центральный пик гистограмм совпадает с полученным средним значением, общий контур по верхам пиков ложится на кривую Гаусса. Отдельно отметим, что доли случаев попадания количества срабатываний в $\sigma$ и $2\sigma$ - окрестности среднего значения близки к теоретической оценке.
	
	Разница между теоретическими оценками и экспериментальными результатами может быть обусловлена погрешностью метода (работа счетчика) и недостаточным количеством проведенных измерений. Притом относительная погрешность определенных величин удовлетворительна (< 2\%). 
\end{document}