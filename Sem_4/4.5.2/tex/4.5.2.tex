\documentclass[12pt,a4paper]{article}
\usepackage[utf8]{inputenc}
\usepackage[english,russian]{babel}
\usepackage{indentfirst}
\usepackage{misccorr}
\usepackage{graphicx}
\usepackage{amsmath}
\usepackage{parskip}
\usepackage[top=2cm, bottom=1cm, left=1cm, right=1cm]{geometry}
\usepackage[most]{tcolorbox}
\usepackage{hyperref}
\usepackage{fancyhdr}

\hypersetup{
	colorlinks,
	citecolor=black,
	filecolor=black,
	linkcolor=blue,
	urlcolor=blue
}

\pagestyle{fancy}
\lhead{Интерференция лазерного излучения}
\rhead{\thepage}
\cfoot{}

\graphicspath{{pic/}, {~/Pictures/TeXImgs/}}

\newcommand{\sfrac}[2]{\dfrac{\strut #1}{\strut #2}}

\newcounter{picture}
\setcounter{picture}{1}
\newcounter{tbl}
\setcounter{tbl}{1}
\newcommand{\embed}[3]{\begin{center}
		\includegraphics[scale=#2]{#1}
		\\\textbf{Рис. \thepicture:} #3
		\label{pic_\thepicture}
		\addtocounter{picture}{1}
\end{center}}
\newcommand{\embedeps}[3]{\begin{center}
		\includegraphics[width=#2\linewidth]{#1}
		\\\textbf{Рис. \thepicture:} #3
		\label{pic_\thepicture}
		\addtocounter{picture}{1}
\end{center}}
\newcommand{\embedtbl}[3]{\begin{center}
		\begin{tabular}{#1}
			#2
		\end{tabular}
		\\\textbf{Табл. \thetbl:} #3
		\label{tbl_\thetbl}
		\addtocounter{tbl}{1}
\end{center}}
\newcommand{\picref}[1]{\hyperref[pic_#1]{Рис. #1}}
\newcommand{\tblref}[1]{\hyperref[tbl_#1]{Табл. #1}}

\begin{document}
	\begin{titlepage}
		\vspace*{\fill}
		
		\begin{center}
			\includegraphics[scale=0.8]{MIPT.png}
			\\[0.7cm]\Huge Московский Физико-Технический Институт\\(национальный исследовательский университет)
			\\[2cm]\LARGE Отчет по эксперименту
			\\[0.5cm]\noindent\rule{\textwidth}{1pt}
			\\\Huge\textbf{Интерференция лазерного излучения}
			\\[-0.5cm]\noindent\rule{\textwidth}{1pt}
		\end{center}
		
		\begin{flushleft}
			\textit{Работа №4.5.2; дата: 04.03.23}\hfill\textit{Семестр: 4}
		\end{flushleft}
		
		\vspace*{\fill}
		
		\begin{flushleft}
			Выполнил: \hspace{\fill} Группа:
			\\Кошелев Александр \hspace{\fill} Б05-105
		\end{flushleft}
	\end{titlepage}
	
	\setcounter{page}{2}
	
	\section{Введение}
	
	\paragraph*{Цель работы:} \hfill
	
	Исследование видности интерференционной картины излучения гелий-неонового лазера и определение длины когерентности излучения.

	\paragraph*{В работе используются:} \hfill
	
	He-Ne-лазер, интерферометр Майкельсона с подвижным зеркалом, фотодиод с усилителем, осциллограф, поляроид, линейка.
	
	\paragraph*{Схема установки:} \hfill
	
	\embed{PIC_1.png}{0.23}{Схема установки}
	
	В работе используется интерферометр Майкельсона. Луч лазера, отражённый от зеркала З и прошедший через параллелепипед Френеля (ПФ), делится делительной призмой ДП на два луча. Первый проходит блок $\text{Б}_1$ с поляроидом $\text{П}_1$ и зеркалом $\text{З}_1$, прикленным к пьезокерамике, которая может совершать малые колебания вдоль луча, с возможность изменения угла наклона зеркала. Второй проходит блок $\text{Б}_2$ с линзой Л, поляроидом $\text{П}_2$ и зеркалом $\text{З}_2$ в фокальной плоскости линзы, чтобы выходящий луч, в отличие от первого, был параллелен входящему. Оба луча, проходя ДП, попадают на сферическое зеркало $\text{З}_3$ и интерферируют на экране. Интенсивность света считывается фотодиодом на осциллограф через щель, параллельную интерфереционным полосам, в центре экрана. На экране осциллографа наблюдаются колебания с изменяющимся периодом, так как на пьезокерамику подаются напряжение, из-за чего её длина колеблется.
	
	\section{Теоретическая справка}
	
	\paragraph*{Гелий-неоновый лазер} \hfill
	
	Лазер представляет собой интерферометр Фабри-Перо -- газовую трубку с двумя параллельными зеркалами по обе стороны. В лазере длиной $L$ для излучения вдоль оси для резонансных частот выполняется:
	
	$$ f_m = \dfrac{c}{\lambda_m} = \dfrac{mc}{2L} $$

	Условие генерации может выполняться для сразу нескольких колебаний с частостами $f_m$, разположенными в диапазоне генерации $2\Delta F$. В этом случае генерируется несколько волн -- \textit{мод} -- межмодовое расстояние для которых:
	
	$$ \Delta \nu = f_{m+1} - f_m = \dfrac{c}{2L} $$
	
	Число мод можно оценить как:
	
	$$ N \approx 1 + \dfrac{2\Delta F}{\Delta \nu} $$
	
	\paragraph*{Видность} \hfill
	
	Видность интерфереционной картины -- параметр, определяемый формулой:
	
	$$ V = \dfrac{I_{\max} - I_{\min}}{I_{\max} + I_{\min}} $$
	
	где $I_{\max}$, $I_{\min}$ -- максимальная и минимальная интенсивности света интерфереционной картины вблизи выбранной точки. Разобьём его на произведение функций параметров установки:
	
	$$ V = V_1 V_2 V_3 $$
	
	Здесь $V_1$ отвечает за соотношение интенсивности интерферирующих волн:
	
	$$ V_1 = \dfrac{2\sqrt{\delta}}{1+\delta} $$
	
	где $\delta = \frac{B_m^2}{A_m^2}$, $A_m$ и $B_m$ -- амплитуды волн. Параметр $\delta$ определяется устройством разделения волн.\\
	Функция $V_2$ отвечает за влияние разности хода и спектрального состава волн

	$$ V_2 = \dfrac{\sum\limits_n A^2_n \cos \frac{2\pi \Delta \nu n l}{c}}{\sum\limits_n A_n^2} $$
	
	где $l$ -- разность хода, $\Delta \nu$ -- спектральный состав излучения, $A_n^2$ -- интенсивности мод. В непрерывном пределе получим:
	
	$$ V_2 = \exp\left(-\frac{\pi \Delta F l}{c}\right) $$
	
	для гауссовой линии излучения с полушириной $\Delta F$ получили гауссову зависимость $V_2 = V_2(l)$ с полушириной 
	
	$$ l_{1/2} = \dfrac{c}{\pi \Delta F}\sqrt{\ln 2} \approx \dfrac{0.26 c}{\Delta F} $$
	
	\embed{PIC_2.png}{0.25}{Зависимость $V_2(l)$}
	
	Последняя функция $V_3$ отвечает за разность в поляризации. Если $\alpha$ -- угол между плоскостями поляризаций волн, то
	
	$$ V_3 = |\cos \alpha| $$
	
	\section{Ход работы}
	
	\subsection{Методика измерений}
	
	Осциллограф мы используем для нахождения следующих величин: фоновой засветки (линия 0 --- перекрыты оба пучка 1 и 2); интенсивность света каждого из пучков (линии 1 или 2 --- перекрыт пучок 2 или 1); максимума и минимума интенсивности интерференционной картины (открыты оба пучка).
	
	\embed{PIC_3.png}{0.17}{Сигнал с фотодиода}
	
	Определенный ранее параметр $\delta$ рассчитывается как:
	
	$$ \delta = \dfrac{h_1}{h_2} $$
	
	При этом видность рассчитывается очевидным образом:
	
	$$ V = \dfrac{h_4 - h_3}{h_4 + h_3} $$
	
	Отсюда мы можем получить компоненты видности, фиксируя одну из них равной единице.
	
	Так, при $ \alpha = 0$ видность $ V_3 = 1 $:
	
	$$ V_2 (l) = \dfrac{V}{V_1} = \dfrac{h_4 - h_3}{h_4 + h_3} \cdot \dfrac{1 + \delta}{2\sqrt{\delta}} $$
	
	А приняв разность хода $ l = 0 $ видность $ V_2 = 1 $:
	
	$$ V_3(\alpha) = \dfrac{V}{V_1} = \dfrac{h_4 - h_3}{h_4 + h_3} \cdot \dfrac{1 + \delta}{2\sqrt{\delta}} $$
	
	\subsection{Исследование зависимости видности от поляризации}
	
	Пронаблюдаем интерференционную картину на экране. Поставим дополнительный поляроид между лазером и ПФ, вращая его, наблюдаем, что поляризация линейная.
	
	\begin{figure}[h]
		\begin{minipage}{0.5\linewidth}
			\embed{PIC_4.png}{0.045, angle=-90}{Минимум интенсивности}
		\end{minipage}
		\begin{minipage}{0.5\linewidth}
			\embed{PIC_5.png}{0.045, angle=-90}{Минимум интенсивности}
		\end{minipage}
	\end{figure}

	Исследуем зависимость видности интерференционной картина от угла $\alpha$ между плоскостями поляризации интерферирущих лучей. На фотодатчике получаем зависимости вида:
	
	\begin{figure}[h]
		\begin{minipage}{0.5\linewidth}
			\embed{PIC_6.png}{0.06}{Показания фотодатчика}
		\end{minipage}
		\begin{minipage}{0.5\linewidth}
			\embed{PIC_7.png}{0.050}{Показания фотодатчика}
		\end{minipage}
	\end{figure}
	
	Полученные данные занесем в \tblref{1}.
	
	\embedtbl{|c|c|c|c|c|c|}{
		\hline
		$\alpha$, $^\circ$ & $h_1$, дел & $h_2$, дел & $h_3$, дел & $h_4$, дел & $V_3$, ед
		\\\hline
		80 $\pm$ 1 & 1.30 $\pm$ 0.05 & 0.20 $\pm$ 0.05 & 1.40 $\pm$ 0.05 & 1.40 $\pm$ 0.05 & 0.00 $\pm$ 0.10
		\\\hline
		75 $\pm$ 1 & 1.30 $\pm$ 0.05 & 0.20 $\pm$ 0.05 & 1.40 $\pm$ 0.05 & 1.60 $\pm$ 0.05 & 0.10 $\pm$ 0.10
		\\\hline
		70 $\pm$ 1 & 1.30 $\pm$ 0.05 & 0.20 $\pm$ 0.05 & 1.20 $\pm$ 0.05 & 1.60 $\pm$ 0.05 & 0.21 $\pm$ 0.11
		\\\hline
		60 $\pm$ 1 & 1.25 $\pm$ 0.05 & 0.20 $\pm$ 0.05 & 1.00 $\pm$ 0.05 & 1.60 $\pm$ 0.05 & 0.33 $\pm$ 0.11
		\\\hline
		50 $\pm$ 1 & 1.20 $\pm$ 0.05 & 0.20 $\pm$ 0.05 & 0.90 $\pm$ 0.05 & 1.80 $\pm$ 0.05 & 0.48 $\pm$ 0.11
		\\\hline
		40 $\pm$ 1 & 1.15 $\pm$ 0.05 & 0.25 $\pm$ 0.05 & 0.70 $\pm$ 0.05 & 1.90 $\pm$ 0.05 & 0.60 $\pm$ 0.10
		\\\hline
		30 $\pm$ 1 & 1.15 $\pm$ 0.05 & 0.30 $\pm$ 0.05 & 0.60 $\pm$ 0.05 & 2.20 $\pm$ 0.05 & 0.71 $\pm$ 0.11
		\\\hline
		20 $\pm$ 1 & 1.15 $\pm$ 0.05 & 0.40 $\pm$ 0.05 & 0.50 $\pm$ 0.05 & 2.40 $\pm$ 0.05 & 0.75 $\pm$ 0.12
		\\\hline
		10 $\pm$ 1 & 1.15 $\pm$ 0.05 & 0.60 $\pm$ 0.05 & 0.40 $\pm$ 0.05 & 2.85 $\pm$ 0.05 & 0.79 $\pm$ 0.14
		\\\hline
		0 $\pm$ 1 & 1.15 $\pm$ 0.05 & 0.60 $\pm$ 0.05 & 0.40 $\pm$ 0.05 & 3.00 $\pm$ 0.05 & 0.81 $\pm$ 0.15
		\\\hline
		-10 $\pm$ 1 & 1.10 $\pm$ 0.05 & 0.80 $\pm$ 0.05 & 0.40 $\pm$ 0.05 & 3.20 $\pm$ 0.05 & 0.79 $\pm$ 0.15
		\\\hline
		-20 $\pm$ 1 & 1.10 $\pm$ 0.05 & 1.00 $\pm$ 0.05 & 0.60 $\pm$ 0.05 & 3.40 $\pm$ 0.05 & 0.70 $\pm$ 0.10
		\\\hline
	}{Зависимость $V_3(\alpha)$}

	Построим теперь линеаризованный график этой зависимости:
	
	\embedeps{PIC_8.eps}{0.75}{График зависимости $V_3(|\cos \alpha|)$}
	
	Таким образом, убеждаемся в верности теоретического предположения.
	
	\subsection{Исследование зависимости видности от разности хода}
	
	Теперь исследуем зависимость видимости интерфереционной картины от разности хода между лучами. Для этого будем перемещать блок $\text{Б}_2$ вдоль направления распространения луча, координата блока $x$ будет определять разность хода. Занесем данные в таблицу:
	
	\embedtbl{|c|c|c|c|c|c|}{
		\hline
		$x$, см & $h_1$, дел & $h_2$, дел & $h_3$, дел & $h_4$, дел & $V_2$, ед
		\\\hline
		8 $\pm$ 0.5 & 1.10 $\pm$ 0.05 & 0.90 $\pm$ 0.05 & 1.00 $\pm$ 0.05 & 2.70 $\pm$ 0.05 & 0.46 $\pm$ 0.08
		\\\hline
		10 $\pm$ 0.5 & 0.65 $\pm$ 0.05 & 1.00 $\pm$ 0.05 & 0.60 $\pm$ 0.05 & 2.40 $\pm$ 0.05 & 0.61 $\pm$ 0.09
		\\\hline
		12 $\pm$ 0.5 & 1.10 $\pm$ 0.05 & 0.90 $\pm$ 0.05 & 0.60 $\pm$ 0.05 & 3.20 $\pm$ 0.05 & 0.69 $\pm$ 0.09
		\\\hline
		17 $\pm$ 0.5 & 0.80 $\pm$ 0.05 & 1.00 $\pm$ 0.05 & 0.40 $\pm$ 0.05 & 1.80 $\pm$ 0.05 & 0.85 $\pm$ 0.20
		\\\hline
		24 $\pm$ 0.5 & 1.00 $\pm$ 0.05 & 1.00 $\pm$ 0.05 & 1.00 $\pm$ 0.05 & 2.40 $\pm$ 0.05 & 0.41 $\pm$ 0.08
		\\\hline
		31 $\pm$ 0.5 & 1.20 $\pm$ 0.05 & 0.90 $\pm$ 0.05 & 1.90 $\pm$ 0.05 & 2.10 $\pm$ 0.05 & 0.05 $\pm$ 0.07
		\\\hline
		33 $\pm$ 0.5 & 1.70 $\pm$ 0.05 & 0.90 $\pm$ 0.05 & 2.40 $\pm$ 0.05 & 2.60 $\pm$ 0.05 & 0.04 $\pm$ 0.07
		\\\hline
		38 $\pm$ 0.5 & 1.20 $\pm$ 0.05 & 0.90 $\pm$ 0.05 & 1.30 $\pm$ 0.05 & 1.60 $\pm$ 0.05 & 0.10 $\pm$ 0.07
		\\\hline
		44 $\pm$ 0.5 & 2.20 $\pm$ 0.05 & 0.90 $\pm$ 0.05 & 1.80 $\pm$ 0.05 & 2.20 $\pm$ 0.05 & 0.11 $\pm$ 0.08
		\\\hline
		45 $\pm$ 0.5 & 1.85 $\pm$ 0.05 & 0.90 $\pm$ 0.05 & 2.20 $\pm$ 0.05 & 2.60 $\pm$ 0.05 & 0.09 $\pm$ 0.08
		\\\hline
		52 $\pm$ 0.5 & 1.80 $\pm$ 0.05 & 0.90 $\pm$ 0.05 & 2.40 $\pm$ 0.05 & 3.20 $\pm$ 0.05 & 0.15 $\pm$ 0.08
		\\\hline
		55 $\pm$ 0.5 & 1.10 $\pm$ 0.05 & 0.90 $\pm$ 0.05 & 1.80 $\pm$ 0.05 & 2.00 $\pm$ 0.05 & 0.05 $\pm$ 0.07
		\\\hline
		59 $\pm$ 0.5 & 2.10 $\pm$ 0.05 & 0.90 $\pm$ 0.05 & 2.60 $\pm$ 0.05 & 3.10 $\pm$ 0.05 & 0.10 $\pm$ 0.08
		\\\hline
		66 $\pm$ 0.5 & 2.00 $\pm$ 0.05 & 0.90 $\pm$ 0.05 & 2.60 $\pm$ 0.05 & 2.80 $\pm$ 0.05 & 0.04 $\pm$ 0.08
		\\\hline
		71 $\pm$ 0.5 & 2.10 $\pm$ 0.05 & 0.90 $\pm$ 0.05 & 2.20 $\pm$ 0.05 & 4.20 $\pm$ 0.05 & 0.34 $\pm$ 0.09
		\\\hline
		73 $\pm$ 0.5 & 2.30 $\pm$ 0.05 & 0.90 $\pm$ 0.05 & 1.80 $\pm$ 0.05 & 4.40 $\pm$ 0.05 & 0.47 $\pm$ 0.09
		\\\hline
		75 $\pm$ 0.5 & 1.60 $\pm$ 0.05 & 0.90 $\pm$ 0.05 & 1.00 $\pm$ 0.05 & 3.60 $\pm$ 0.05 & 0.59 $\pm$ 0.09
		\\\hline
		80 $\pm$ 0.5 & 1.40 $\pm$ 0.05 & 0.90 $\pm$ 0.05 & 0.60 $\pm$ 0.05 & 3.60 $\pm$ 0.05 & 0.73 $\pm$ 0.09
		\\\hline
		
	}{Зависимость $V_2(x)$}
	
	Построим график данной зависимости:
	
	\embedeps{PIC_9.eps}{0.75}{График зависимости $V_2(x)$}
	
	Вертикальными линиями отмечены положения максимумов $x_1 = (15 \pm 1)$ см, $x_2 = (80 \pm 1)$ см. Тогда расстояние между зеркалами лазера:
	
	$$ L = \sfrac{x_2 - x_1}{2} = (32 \pm 1)\, \text{см} $$
	
	Межмодовое расстояние:
	
	$$ \Delta \nu = \dfrac{c}{2L} = (4.8 \pm 0.2) \cdot 10^8\, \text{Гц} $$
	
	Полуширина кривой из графика:
	
	$$	l_{1/2} \approx 10 \pm 2 \text{ см}$$
	
	Откуда получаем полную ширину спектра:
	
	$$ \Delta F = \dfrac{0.26 c}{l_{1/2}} = (7.8 \pm 1.6) \cdot 10^8 \text{ Гц} $$
	
	А число мод равно:
	
	$$ N = 1 + \dfrac{2\Delta F}{\Delta \nu} = 4 \pm 1 $$
	
	\newpage
	
	Из соотношения неопределенностей получим также длину когерентности:
	
	$$ L_{c} \sim \sfrac{c}{\Delta F} \sim 30\, \text{см}$$
	
	\section{Выводы}
	
	\begin{itemize}
		\item Проверена теоретически предсказанная зависимость видности интерференционной картины от поляризации
		\item Проверена теоретически предсказанная зависимость видности интерференционной картины от разности хода
		\item Определено расстояние между зеркалами лазера:
		$$ L = (32 \pm 1)\, \text{см}$$
		\item Определено межмодовое расстояние для используемого лазера:
		$$ \Delta \nu = (4.8 \pm 0.2) \cdot 10^8\, \text{Гц}$$
		\item Определена полная ширина спектра:
		$$ \Delta F = (7.8 \pm 1.6) \cdot 10^8 \text{ Гц} $$
		\item Оценено число мод излучения лазера:
		$$ N = 4 \pm 1 $$
		\item Оценена длина когерентности лазерного излучения:
		$$ L_{c} \sim 30\, \text{см}$$
	\end{itemize}
	
\end{document}