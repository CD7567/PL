\documentclass[12pt,a4paper]{article}
\usepackage[utf8]{inputenc}
\usepackage[english,russian]{babel}
\usepackage{indentfirst}
\usepackage{misccorr}
\usepackage{graphicx}
\usepackage{amsmath}
\usepackage{parskip}
\usepackage[top=2cm, bottom=1cm, left=1cm, right=1cm]{geometry}
\usepackage[most]{tcolorbox}
\usepackage{hyperref}
\usepackage{fancyhdr}

\hypersetup{
	colorlinks,
	citecolor=black,
	filecolor=black,
	linkcolor=blue,
	urlcolor=blue
}

\pagestyle{fancy}
\lhead{Эффект Поккельса}
\rhead{\thepage}
\cfoot{}

\graphicspath{{pic/}, {~/Pictures/TeXImgs/}}

\newcommand{\sfrac}[2]{\dfrac{\strut #1}{\strut #2}}

\newcounter{picture}
\setcounter{picture}{1}
\newcounter{tbl}
\setcounter{tbl}{1}
\newcommand{\embed}[3]{\begin{center}
		\includegraphics[scale=#2]{#1}
		\\\textbf{Рис. \thepicture:} #3
		\label{pic_\thepicture}
		\addtocounter{picture}{1}
\end{center}}
\newcommand{\embedeps}[3]{\begin{center}
		\includegraphics[width=#2\linewidth]{#1}
		\\\textbf{Рис. \thepicture:} #3
		\label{pic_\thepicture}
		\addtocounter{picture}{1}
\end{center}}
\newcommand{\embedtbl}[3]{\begin{center}
		\begin{tabular}{#1}
			#2
		\end{tabular}
		\\\textbf{Табл. \thetbl:} #3
		\label{tbl_\thetbl}
		\addtocounter{tbl}{1}
\end{center}}
\newcommand{\picref}[1]{\hyperref[pic_#1]{Рис. #1}}
\newcommand{\tblref}[1]{\hyperref[tbl_#1]{Табл. #1}}

\begin{document}
	\begin{titlepage}
		\vspace*{\fill}
		
		\begin{center}
			\includegraphics[scale=0.8]{MIPT.png}
			\\[0.7cm]\Huge Московский Физико-Технический Институт\\(национальный исследовательский университет)
			\\[2cm]\LARGE Отчет по эксперименту
			\\[0.5cm]\noindent\rule{\textwidth}{1pt}
			\\\Huge\textbf{Эффект Поккельса}
			\\[-0.5cm]\noindent\rule{\textwidth}{1pt}
		\end{center}
		
		\begin{flushleft}
			\textit{Работа №4.7.2; дата: 11.03.23}\hfill\textit{Семестр: 4}
		\end{flushleft}
		
		\vspace*{\fill}
		
		\begin{flushleft}
			Выполнил: \hspace{\fill} Группа:
			\\Кошелев Александр \hspace{\fill} Б05-105
		\end{flushleft}
	\end{titlepage}

	\setcounter{page}{2}

	\section{Введение}
	
	\paragraph*{Цель работы:} \hfill
	
	Исследовать интерференцию рассеянного света, прошедшего кристалл; наблюдать изменение характера поляризации света при наложении на кристалл электрического поля.
	
	\paragraph*{В работе используются:} \hfill
	
	Гелий-неоновый лазер, поляризатор, кристалл ниобата лития, матовая пластинка, экран, источник высоковольтного переменного и постоянного напряжения, фотодиод, осциллограф, линейка.
	
	\paragraph*{Схема установки:} \hfill
	
	\embed{PIC_1.png}{0.2}{Схема установки}
	
	\section{Теоретические сведения}
	
	Рассмотрим кристалл ниобата лития: его оптические свойства обладают симметрией вращения относительно выделенного направления -- оптической оси $\mathrm{o}Z$. Для волны, распространяющейся вдоль $\mathrm{o}Z$, показатель преломления равен $n_o$, а для волны, перпендикулярной оптической оси, $n_e$, причем для ниобата лития $n_o > n_e$.
	
	Волну длины $\lambda = 2\pi/k$, проходящую под углом $\theta$ к оси $\mathrm{o}Z$ в кристалле, раскладывают на обыкновенную и необыкновенную. Для вектора напряженности обыкновенной волны верно: $\vec{E_o} \parallel \vec{k} \times \vec{e_z}$, и показатель преломления $n_1 = n_o$. Для вектора напряженности необыкновенной: $\vec{E_e} \perp \vec{k} \times \vec{e_z}$, и показатель преломления $n_2$ зависит от $\theta$ по закону:
	
	$$ \frac{1}{n_2^2} = \frac{\cos^2\theta}{n_o^2} + \frac{\sin^2\theta}{n_e^2} $$
	
	Сдвиг фаз обыкновенной и необыкновенной волн при прохождении кристалла длиной $l$ составляет: 
	
	$$ \Delta = kl(n_1 - n_2) = \frac{2\pi}{\lambda} \cdot l (n_1 - n_2) $$
	
	С учетом зависимости $n_2(\theta)$ для малых углов $\theta$ в приближении $n_o \approx n_e$: 
	
	$$ \Delta = \frac{2\pi}{\lambda}\cdot l (n_o - n_e)\theta^2 $$
	
	Направления постоянной разности фаз задают конусы $\theta = \mathrm{const}$, следовательно, интерференционная картина представляет собой концентрические окружности.
	
	\embed{PIC_2.png}{0.2}{Главные направления при наложении электрического поля}
	
	Поместим кристалл ниобата лития в постоянное электрическое поле $\vec{E}_{ext}$, направленное по оси $\mathrm{o}X$, перпендикулярной оптической оси $\mathrm{o}Z$. В плоскости $\mathrm{o}XY$ возникают быстрая и медленная оси под углами $45^\circ$ к $\mathrm{o}X$, $\mathrm{o}Y$, соответствующие показателям преломления $(n_o - \Delta n)$ и $(n_o + \Delta n)$, здесь $\Delta n = A\cdot E_{ext}$, $A$ - константа, зависящая от свойств материала. В этом и заключается \textit{эффект Поккельса}. 
	
	Появление главных направлений $\xi$ и $\eta$ иллюстрирует \picref{2}.
	
	\section{Ход работы}
	
	\subsection{Исследование интерференции рассеянного света}
	
	Схема наблюдения интерференционной картины приведена на \picref{3}. Свет лазера, поляризованный в вертикальной плоскости, рассеивается на матовой пластинке и проходит через двоякопреломляющий кристалл. На выходе из кристалла стоит поляроид. Параметры установки: размеры кристалла $3 \times 3 \times 26$ mm, длина волны гелий-неонового лазера $\lambda = 630$ nm, показатель преломления $n_o = 2.29$, расстояние до экрана от центра кристалла $L = (78 \pm 1)$ sm.
	
	\embed{PIC_3.png}{0.2}{Схема наблюдения интерференционной картины}
	
	Интерференционная картина, создаваемая обыкновенной и необыкновенной волнами, наблюдается в скрещенной поляризации. Для луча, идущего вдоль оптической оси $Z$, верно: $n_o = n_e$; его поляризация не изменяется в кристалле, луч не проходит через анализатор, и в центре интерференционной картины находится темное пятно. Следующий минимум интенсивности соответствует сдвигу фаз между волнами на $2\pi$, поэтому условие на $m$-ое темное кольцо запишется с использованием формулы в виде:
	
	$$ \Delta = 2\pi m \Longleftrightarrow \theta_m^2 = \frac{m\lambda}{l(n_o - n_e)} $$
	
	Здесь $l = 26$ мм -- длина кристалла вдоль оптической оси.
	
	По закону Снеллиуса, угол преломления на внешней границе кристалла: $\theta_{ex} = n_o\theta$. Тогда для радиуса $m$-ого темного кольца $r_m$ верно: 
	
	$$ r_m^2 = \frac{\lambda}{l} \frac{(n_oL)^2}{(n_o - n_e)} \cdot m $$
	
	\embed{PIC_4.png}{0.1}{Интерференционная картина}
	
	Снимем зависимость радиусов темных концентрических колец от номера максимума $r_m(m)$. Погрешность величины $r_m$ примем равной $0.5$ см из-за расплывчатости картинки и неточности прямых измерений. Результаты измерений занесены в таблицу \tblref{1}. 
	
	\embedtbl{|c|c|c|c|c|c|}{
		\hline
		$m$ & 1 & 2 & 3 & 4 & 5
		\\\hline
		$r_m$, см & 2.8 & 4.0 & 5.0 & 5.7 & 6.5
		\\\hline
	}{Зависимость $r_m(m)$}

	Построим график линеаризованной зависимости:
	
	\embedeps{PIC_5.eps}{0.6}{График зависимости $r_m^2(m)$}
	
	Из графика получаем коэффициент наклона $\gamma$ и двулучепреломление ниобата лития:
	
	$$ \gamma = (8.53 \pm 0.18)\, \text{см}^2 \implies n_o - n_e = (0.086 \pm 0.002)$$
	
	\subsection{Изменение характера поляризации света при наличии внешнего поля}
	
	При наложении электрического поля в кристалле возникают быстрая ось $\xi$ и медленная ось $\eta$, изображенные на рисунке \picref{2}; разложим вектор напряженности волны по ним. После прохождения кристалла разность фаз между $E_\eta$ и $E_\xi$ составит $\Delta = \frac{2\pi}{\lambda} \cdot2l\Delta n = \frac{4\pi}{\lambda} \frac{l}{d} AU$, где $U = E_{ext}d$ -- напряжение на кристалле, $d = 3$ мм -- его поперечный размер, $l = 26$ мм -- длина пути луча. Поляроид пропускает горизонтальную составляющую волны. Значит, выходная напряженность складывается из проекций $E_\eta$ и $E_\xi$ на ось $\mathrm{o}X$:
	
	$$ E = \frac{E_0}{2} \cdot e^{i(\omega t - kl)} (e^{i\Delta/2} - e^{i\Delta/2}) = \frac{E_0}{2} \cdot  e^{i(\omega t - kl + \pi/2)} \sin\frac{\Delta}{2} $$
	
	Здесь $E_0$ - амплитуда входной волны.
	
	Отсюда интенсивность выходной волны: 
	
	$$ I = I_0 \sin^2\frac{\Delta}{2} = I_0\sin^2 \left(\frac{\pi}{2}\frac{U}{U_{\lambda/2}}\right) $$
	
	Здесь введено \textbf{полуволновое напряжение} $U_{\lambda/2} = \frac{\lambda}{4A}\frac{d}{l}$, соответствующее максимальной интенсивности на выходе.
	
	При параллельных поляризациях лазера и анализатора получаем следующую зависимость $I(U)$:
	
	$$ I = I_0 \cos^2\left(\frac{\pi}{2}\frac{U}{U_{\lambda/2}}\right) $$
	
	Схема установки, включающая блок питания, фотодиод и осциллограф, используемые в этой части работы, показана на \picref{1}.  
	
	Для скрещенных поляризаций при напряжениях $U = (2k - 1)U_{\lambda/2}$ наблюдается максимум интенсивности, при $U = 2kU_{\lambda/2}$ -- минимум. Для параллельных поляризаций ситуация противоположная.
	
	Напряжения, соответствующие последовательным экстремумам интенсивности для разных поляризаций, содержатся в \tblref{2}.	В 100 делениях шкалы блока питания 1.5 kV. 
	
	\embedtbl{|c|c|c|}{
		\hline
		& Скрещенные поляризации & Параллельные поляризации
		\\\hline
		$U_{\lambda/2}$, ед & 27 $\pm$ 1 (max) & 26 $\pm$ 1 (min)
		\\\hline
		$2U_{\lambda/2}$, ед & 52 $\pm$ 1 (min) & 52 $\pm$ 1 (max)
		\\\hline
		$3U_{\lambda/2}$, ед & 79 $\pm$ 1 (max) & 78 $\pm$ 1 (min)
		\\\hline
	
	}{Последовательные экстремумы интенсивности}
	
	\begin{figure}[h]
		\begin{minipage}{0.5\linewidth}
			\embed{PIC_6.png}{0.15}{Минимум интенсивности}
		\end{minipage}
		\begin{minipage}{0.5\linewidth}
			\embed{PIC_7.png}{0.15}{Максимум интенсивности}
		\end{minipage}
	\end{figure}
	
	По таблице \tblref{2} найдем среднее значение полуволнового напряжения: 
	
	$$ U_{\lambda/2} = (390 \pm 15)\,\text{В} $$ 
	
	При напряжении $U_{\lambda/4}$ интенсивности при скрещенной и параллельной поляризациях совпадают. Выставим экспериментальное значение напряжения $U_{\lambda/4} = U_{\lambda/2}/2 \approx 230$ В. При вращении анализатора интенсивность наблюдаемого пятна практически не меняется, что свидетельствует о круговой поляризации и подтверждает правильность расчетов.  
	
	Подключим фотодиод к $Y$-входу осциллографа. На $X$-вход подадим переменное напряжение с блока питания. В режиме DUAL на экране осциллографа получаются фигуры Лиссажу, отвечающие зависимости $I(U)$. Она задается формулами в начале раздела; для скрещенных поляризаций имеет вид синусоиды, взятой на симметричном отрезке, а для параллельных поляризаций представляет собой косинусоиду. Таким образом, фигуры Лиссажу для разных поляризаций при одинаковом значении амплитуды напряжения $U$ отличаются по фазе на $\pi/2$.
	
	\begin{figure}[h]
		\begin{minipage}{0.33\linewidth}
			\embed{PIC_8.png}{0.08}{$U = U_{\lambda/2}$ при $\parallel$}
		\end{minipage}
		\begin{minipage}{0.33\linewidth}
			\embed{PIC_9.png}{0.05426}{$U = 2U_{\lambda/2}$ при $\parallel$}
		\end{minipage}
		\begin{minipage}{0.33\linewidth}
			\embed{PIC_10.png}{0.07}{$U = 3U_{\lambda/2}$ при $\parallel$}
		\end{minipage}
	\end{figure}
	
	Полуволновое напряжение, определенное по фигурам Лиссажу в точности совпадает с рассчетным.
	
	\section{Выводы}
	
	\begin{itemize}
		\item Исследована интерференция лазерного излучения после матовой пластинки на одноосном кристалле.
		\item Определено двулучепреломление кристалла ниобата лития на длине волны $\lambda = 630$ нм
		$$ n_o - n_e = (0.086 \pm 0.002) $$
		Значение в пределах погрешности совпадает с табличным $(n_o - n_e)_{\mathrm{ref}} = 0.084$
		\item Для данного кристалла определено полуволновое напряжение
		$$ U_{\lambda/2} = (390 \pm 15)\, \text{В} $$
		Это значение подтверждено наблюдением соответствующих фигур Лиссажу.
	\end{itemize}
\end{document}